\documentclass{beamer}

%\usetheme{Warsaw}
\usetheme{Frankfurt}
\beamertemplatenavigationsymbolsempty

\usepackage[utf8]{inputenc}
\usepackage[francais]{babel}
\usepackage{hyperref}
\usepackage{amsmath}
\usepackage{graphicx}
\usepackage{sidecap}
\usepackage{caption}
\graphicspath{{./img/}}
\DeclareGraphicsExtensions{.png, .jpeg, .jpg}


\usepackage{listings}
\usepackage{setspace} 
\usepackage{url}


\renewcommand*\thesection{\arabic{section}}


\AtBeginSection[]{%
  \begin{frame}<beamer>
    \frametitle{Outline}
    \tableofcontents[sectionstyle=show/hide,subsectionstyle=hide/show/hide]
  \end{frame}
  \addtocounter{framenumber}{-1}
}

% Custom colors
\usepackage{color}


\definecolor{Code}{rgb}{0,0,0}
\definecolor{Decorators}{rgb}{0.5,0.5,0.5}
\definecolor{Numbers}{rgb}{0.5,0,0}
\definecolor{MatchingBrackets}{rgb}{0.25,0.5,0.5}
\definecolor{Keywords}{rgb}{0,0,1}
\definecolor{self}{rgb}{0,0,0}
\definecolor{Strings}{rgb}{0,0.63,0}
\definecolor{Comments}{rgb}{0,0.63,1}
\definecolor{Backquotes}{rgb}{0,0,0}
\definecolor{Classname}{rgb}{0,0,0}
\definecolor{FunctionName}{rgb}{0,0,0}
\definecolor{Operators}{rgb}{0,0,0}
\definecolor{Background}{rgb}{0.98,0.98,0.98}

\lstnewenvironment{python}[1][]{
\lstset{
numbers=left,
numberstyle=\footnotesize,
numbersep=1em,
xleftmargin=1em,
framextopmargin=2em,
framexbottommargin=2em,
showspaces=false,
showtabs=false,
showstringspaces=false,
frame=l,
tabsize=4,
% Basic
basicstyle=\ttfamily\small\setstretch{1},
backgroundcolor=\color{Background},
language=Python,
% Comments
commentstyle=\color{Comments}\slshape,
% Strings
stringstyle=\color{Strings},
morecomment=[s][\color{Strings}]{"""}{"""},
morecomment=[s][\color{Strings}]{'''}{'''},
% keywords
morekeywords={import,from,class,def,for,while,if,is,in,elif,else,not,and,or,print,break,continue,return,True,False,None,access,as,,del,except,exec,finally,global,import,lambda,pass,print,raise,try,assert},
keywordstyle={\color{Keywords}\bfseries},
% additional keywords
morekeywords={[2]@invariant},
keywordstyle={[2]\color{Decorators}\slshape},
emph={self},
emphstyle={\color{self}\slshape},
%
}}{}



\title{Python : Une Introduction}
\institute{ASCII}
\author{Rémi Bois}
\date{\today}

\begin{document}

\begin{frame}
  \maketitle
\end{frame}

\section{Introduction}
\label{sec:intro}


\begin{frame}
  \frametitle{Objectifs de la présentation}

  \begin{block}{A faire}
    \begin{itemize}
    \item Présenter le language
    \item Présenter un environnement de travail pour le language
      (virtualenv, emacs for python, pip, ...)
    \item Donner des pointeurs vers des ressources pour aller plus loin
    \end{itemize}
  \end{block}

  \begin{block}{On va éviter}
    \begin{itemize}
    \item Affirmer que c'est mieux que
      [insérez-votre-language-préféré]

    \item Être trop technique
    \item Être trop simple
    \end{itemize}
  \end{block}
  
% Do-s & Don't-s
\end{frame}

\begin{frame}
  \frametitle{Python, Kézako ?}

  
  \begin{block}{Caractéristiques du language}
    \begin{itemize}
    \item Python est un language de script datant de 1989. Successeur du
      language ABC, lui même inspiré de ... SETL (!)
    \item Python 2 en 2000, Python 3 en 2008
    \item Il est interprété (et pas compilé).
    \item Il propose une programmation : impérative, fonctionnelle,
      object, procédurale.
    \item High level :  "A programming language is low level when its
      programs require attention to the irrelevant." Alan J. Perlis,
      Premier Prix Turing.
    \end{itemize}
  \end{block}

  \begin{block}{Son origine}
    \begin{itemize}
    \item Nommé après les Monty Python
    \item Parenté avec Perl ``Many ways to do it'' $\rightarrow$ ``One
      obvious way to do it''
    \end{itemize}
  \end{block}
  
% langage de programmation/script
% date d'apparition
% parenté avec les Monthy Python
% Parenté avec Perl "many ways to do it" -> "one obvious way"
\end{frame}

\begin{frame}
  \frametitle{Un guide}

  \begin{block}{BDFL}
    Il bénéficie d'un BDFL (Benevolent Dictator for Life) : Guido van
    Rossum.  
  \end{block}
  
  \begin{figure}[H]
    \centering
    \includegraphics[height=0.3\textwidth]{guido}
    \caption{Guido van Rossum, BDFL}
    \label{fig:guido}
  \end{figure}

% Guido

\end{frame}

\begin{frame}
  \frametitle{Popularité}

  \begin{block}{Un language fait pour l'open source}
    \begin{itemize}
    \item Multi-plateforme
    \item Interprété (le code est donc distribué tel quel)
    \item Très présent sur Github
    \end{itemize}
    
  \end{block}

  \begin{figure}[H]
    \centering
    \includegraphics[height=0.33\textwidth]{trendingpython}
    \caption{Un langage qui monte qui monte...}
    \caption*{\tiny{Source : redmonk : \url{http://redmonk.com/sogrady/2014/01/22/language-rankings-1-14/}}}
    \label{fig:trendingpython}
  \end{figure}

% Place dans l'open source

\end{frame}

\begin{frame}
  \frametitle{Des gros projets}

  \begin{block}{Quelques exemples}
    \begin{itemize}
    \item Django, un framework web en python
    \item Dropbox, pour partager ses fichiers
    \item Civilization IV
    \item Répandu chez Google, la NASA, EDF, ...
    \end{itemize}

  \end{block}
    
% Grands projets (voir django, ...)
\end{frame}

\begin{frame}
  \frametitle{Une bonne réputation}

  \begin{figure}[H]
    \centering
    \includegraphics[height=0.6\textwidth]{xkcd}
    \caption{xkcd 353}
    \label{fig:xkcd}
  \end{figure}
  \centering
  \tiny{Source : xkcd : \url{http://xkcd.com/353/}}

% c'est bien ... http://xkcd.com/353/
\end{frame}


\begin{frame}
  \frametitle{Sa particularité ?}
  
  \begin{block}{Import this}
    \begin{itemize}
    \item Des guidelines claires, documentées (PEP 8)
    \item Une philosophie
    \end{itemize}
  \end{block}

  \begin{block}{Un langage pour spécialistes}
    \begin{itemize}
    \item NLTK
    \item NumPy
    \end{itemize}
  \end{block}

  \begin{block}{Adossé au C}
    \begin{itemize}
    \item et donc rapide (contrairement à une idée répandue)
    \end{itemize}
  \end{block}

% Lisibilité (quote import this)
% Très bon dans certains domaines (nltk, numPy, ...)
% Adossé au C (ahah)
\end{frame}

\begin{frame}
  \frametitle{Une version 3 pas encore tout à fait adoptée ?}

  \begin{block}{Py2 vs Py3}
    \begin{itemize}
    \item Des modules pas tous passés à Python 3
    \item Un portage non trivial (malgré quelques scripts comme py2to3)
    \end{itemize}
  \end{block}

  \begin{block}{Python 3 est le futur}
    \begin{itemize}
    \item Python 3 résoud beaucoup de problèmes sémantiques / stylistiques.
    \item La plupart des modules prévoient un portage Python 3.
    \item Le développement de Python 2 est arrêté (Python 2.7 est la
      dernière version).
    \end{itemize}
  \end{block}
% Python 2 vs Python 3
% problème des modules non portés
% améliorations (/, print, ...)
\end{frame}

\section{Le Langage}
\label{sec:language}

\begin{frame}[fragile]
  \frametitle{L'avènement des tabulations}

  \begin{block}{Donald Knuth l'avait prédit}
    \begin{quote}
      We will perhaps eventually be writing only small modules which are
      identified by name as they are used to build larger ones, so that
      devices like indentation, rather than delimiters, might become
      feasible for expressing local structure in the source language.   
    \end{quote}
    \centering
    \tiny
    — Donald E. Knuth, in "Structured Programming with goto
    Statements" Dec. 1974
  \end{block}


% Objectif (lisibilité)
 % We will perhaps eventually be writing only small modules which are
 % identified by name as they are used to build larger ones, so that
 % devices like indentation, rather than delimiters, might become
 % feasible for expressing local structure in the source language. 
 %    — Donald E. Knuth, in "Structured Programming with goto
 %    Statements", Computing Surveys, Vol 6 No 4, Dec. 1974  
% Un exemple Java/C++ -> Python
\end{frame}



\begin{frame}[fragile]
  \frametitle{Le classique Hello World}

    \begin{block}{Python}
    \begin{python}
  print("Hello World")
    \end{python}

    \end{block}

    \begin{block}{VS Java...}
    \begin{lstlisting}[language=Java]
  public class HelloWorld {
    public static void main(String[] args){
      System.out.println("Hello, World");
    }
  }
    \end{lstlisting}
  \end{block}
\end{frame}


\begin{frame}[fragile]
  \frametitle{Hello World, version longue}

  \begin{block}{Python}
    \begin{python}
  def main():
    print("Hello World")

  if __name__ == "__main__":
    main()
    \end{python}

    \end{block}
\end{frame}

\begin{frame}[fragile]
  \frametitle{Le nommage en Python}


  \begin{block}{Les variables}
    \begin{python}
  a = 5 # pas de typage
  b = 3.0
  a = [] # reassigne une valeur a un nom
  b = "toto" # chaine de caractere
  a = 't' # caractere
  b = a # b = 't'
    \end{python}
  \end{block}

% Non verbosité
% nommage des variables plutôt que typage
% exemples
\end{frame}


\begin{frame}[fragile]
  \frametitle{Les fonctions}

  \begin{block}{En C++}
    \begin{python}
void* foo(std::vector<int>* first,
           bar& second, baz&& other)
  {
   double length = other.length;
   return (*other)->length;
 }
    \end{python}
  \end{block}


  \begin{block}{En python}

    \begin{python}
def nomDeFonction(foo, bar, optionnel=None):
    # tambouille
    return result

toto = nomDeFonction("a" , 5)
    \end{python}
  \end{block}

\end{frame}

\begin{frame}[fragile]
  \frametitle{Les conditions et les boucles}

  \begin{block}{Les Ifs}
    \begin{python}
if cond1 and cond2:
  print("We made it !")
elif cond1:
  print("Only cond1 !")
else:
  print("None is True !")
    \end{python}
  \end{block}

  \begin{block}{While, la boucle la plus rare}
    \begin{python}
i = 0
while i < 10:
  i += 1
  print(i)
    \end{python}
  \end{block}

% If, While, For
\end{frame}


\begin{frame}[fragile]
  \frametitle{For, le while done right}

  \begin{block}{Un for classique}
    \begin{python}
for i in range(5):
  print(i) # 0, 1, 2, 3, 4
    \end{python}
  \end{block}


  \begin{block}{Itérer avec un for}
    \begin{python}
for item in maliste:
  print(item)
    \end{python}
  \end{block}

\end{frame}

\begin{frame}[fragile]
  \frametitle{Les tuples présents, comme dans tout bon langage}

  \centering

  Toujours pas présents en Java...

  \begin{block}{Paires, triplets, ...}
    \begin{python}
paire = (17, "Mars")
triplet = ('a', "sc", "ii")
print(triplet[0] + triplet[1] + triplet[2])
# ascii
    \end{python}
  \end{block}

  \begin{block}{Swap, unpacking}
    \begin{python}
a, b = ("at", "al")
print(a + b) # atal
print(a, b) # at al
a, b = b, a
print(a + b) # alat
    \end{python}
  \end{block}

% taunt Java
% définition, accès, swap de variables...
\end{frame}

\begin{frame}[fragile]
  \frametitle{Les listes comme on les connaît}

  \begin{block}{Création et accès}
    \begin{python}
maliste = [1, 2, "toto", 5.0]
print(maliste[0]) # 1
print(maliste[len(maliste) - 1]) # 5.0
print(maliste[-1]) # 5.0
    \end{python}
  \end{block}

  \begin{block}{Un parcours en for}
    \begin{python}
# bien
for item in maliste:
  print(item)

# pas bien
for i in range(len(maliste)):
  print(maliste[i])
    \end{python}
  \end{block}

% définition, parcourt, fonctions principales
\end{frame}


\begin{frame}[fragile]
  \frametitle{Les slices}
  \begin{block}{Of slice and men}
    \begin{python}
maliste = [1, "deux", 3.3]

print(maliste[0:2]) # [1, "deux"]
print(maliste[0:-1]) # [1, "deux"]
print(maliste[:2]) # [1, "deux"]

print(maliste[1:]) # ["deux", 3.3]

a, b = maliste[0:2]
print(a, b) # 1 deux
    \end{python}
  \end{block}
\end{frame}

\begin{frame}[fragile]
  \frametitle{List comprehension}

  \begin{block}{Un Python intelligent}
    \begin{python}
l1 = [1, 2, 3]
l2 = [4, 5, 6]

print(zip(l1, l2)) # [(1, 4), (2, 5), (3, 6)]

maliste = [a + b for a, b in zip(l1, l2)]
print(maliste) # [5, 7, 9]
    \end{python}
  \end{block}

  \begin{block}{Un outil puissant}
    \small
    \begin{python}
maliste = [1, 2, 3, 4, 5]

even = [number for number in maliste
               if number % 2 == 0]
print(even) # [2, 4]
    \end{python}
    
  \end{block}
% exemple simple avec zip (en lien avec les tuples)
\end{frame}

\begin{frame}[fragile]
  \frametitle{List comprehension avancée}

  \begin{block}{Nested Lists}
    \begin{python}
maliste = [1, 2, 3, 4]
[x+1 for x in [y+1 for y in maliste]] 
# [3, 4, 5, 6]
    \end{python}
  \end{block}

  \begin{block}{Avec des fonctions}
    \begin{python}
maliste = [1, 2, 3, 4]
[plus_un(x) for x in maliste] # [2, 3, 4, 5]
    \end{python}
  \end{block}

% nested lists, fonction sur les éléments
\end{frame}

\begin{frame}[fragile]
  \frametitle{Programmer avec class}

  \begin{block}{Définition d'une classe}
    \begin{python}
class MaClasse():
  attribut = 5

  def plusGrand(self, val):
    return self.attribut > val

instance = MaClasse()
print(instance.attribut) # 5
print(instance.plusGrand(7)) # False
    \end{python}
  \end{block}
% class, héritage méthodes
\end{frame}


\begin{frame}[fragile]
  \frametitle{Héritage}

  \begin{block}{Hériter de object}
    \begin{python}
class BaseClass(object):
  def __init__(self, attval):
    self.attribut = attval
inst = BaseClass(5)
print(inst.attribut) # 5
    \end{python}

  \end{block}

  \begin{block}{Un riche héritage}
  \begin{python}
class MultiClass(BaseClass, AutreClasse):
  attr = 5
  \end{python}
    
  \end{block}
\end{frame}

\begin{frame}
  \frametitle{We're all consenting adults}

  \begin{block}{Pivate, Package, Public, Protected, DontTouchIt, ...}
    \begin{itemize}
    \item On estime que les développeurs sont en droit d'accéder aux
      variables
    \item On peut simuler une restriction via des décorateurs
    \item On utilise une convention de nommage (\_mavar)
    \end{itemize}
  \end{block}

  \begin{block}{Une histoire amusante}
  Un groupe de pirates menace de mettre à disposition plus de 5.000
  projets open-source java, en mettant tous les attributs et méthodes
  en public !

  \url{http://steve-yegge.blogspot.fr/2010/07/wikileaks-to-leak-5000-open-source-java.html}
  \end{block}
% méthodes / attributs privés
% taunt Java with Wikileaks
% http://steve-yegge.blogspot.fr/2010/07/wikileaks-to-leak-5000-open-source-java.html

\end{frame}

\begin{frame}[fragile]
  \frametitle{Les maps, filter, ... reduce ?}

  \begin{block}{Des fonctions standard}
    \begin{itemize}
    \item Python propose les deux fonctions de base
    \item Python a supprimé reduce() dans sa version 3
    \end{itemize}
  \end{block}

  \begin{block}{Plutôt inutiles...}
    \begin{python}
l = [1, 2, 3]

map(str, l) # ['1', '2', '3']
[str(x) for x in l] # ['1', '2', '3']

filter(l, greaterThan2) # [2, 3]
[x for x in l if x >= 2] # [2, 3]
    \end{python}
  \end{block}
% fonctions avec exemple
\end{frame}

\begin{frame}[fragile]
  \frametitle{Les lambda expressions}
  \begin{block}{Définition locale d'une fonction}
    \begin{python}
l = [1, 2, 3]

map(lambda x: x+1, l) # [2, 3, 4]
    \end{python}
  \end{block}

  \begin{block}{Utile pour}
    \begin{itemize}
    \item Les map et les filter
    \item Les fonctions prenant une fonction en paramètre (sort)
    \end{itemize}
  \end{block}

% exemple
\end{frame}

\begin{frame}[fragile]
  \frametitle{Les exceptions}

  \begin{block}{Better ask for forgiveness than for permission}
    \begin{python}
x = None
try:
  x = 5 / 0
except:
  x = 5
    \end{python}
  \end{block}

  \begin{block}{La lecture de fichiers}
    \begin{python}
filename = "./fichier.txt"
with open(filename, 'r') as monfichier:
  for line in monfichier.readlines():
    print(line)
    \end{python}
  \end{block}

% Better ask for forgiveness than for permission
% exemple
\end{frame}

\begin{frame}[fragile]
  \frametitle{Breaks \& Continues}

  \begin{block}{Pour arrêter les boucles}
    \begin{python}
for i in range(5):
  res = i
  if i == 3:
    break
print(res) # 3
    \end{python}
  \end{block}

  \begin{block}{Pour passer son tour}
    \begin{python}
for i in range(4):
  if i == 0:
    continue
  print(5/i)
    \end{python}
  \end{block}

% exemple
\end{frame}

\begin{frame}[fragile]
  \frametitle{On s'amuse avec les fonctions}

  \begin{block}{Des listes et des fonctions}
    \begin{python}
def plusun(x):
  return x+1

def plusdeux(x):
  return x+2

fun = [plusun, plusdeux]
val = [1, 2, 3]
[[f(x) for x in val] for f in fun]
# [[2, 3, 4], [3, 4, 5]]
    \end{python}
  \end{block}
\end{frame}

\begin{frame}
  \frametitle{Et encore quelques autres choses}

  \begin{block}{Le reste, en bonus}
    \begin{itemize}
    \item Les décorateurs
    \item Les générateurs, itérateurs, yield statement
    \item Les ensembles (sets) et leurs opérateurs
    \item Les indispensables codecs/io, os, sys, re
    \item None, is, == et les Booléens
    \item itertools, datetime, copy, print, pickle
    \end{itemize}
  \end{block}
\end{frame}

\section{Environnement et Bonnes Pratiques}
\label{sec:environment}
% emacs, virtualenv, pip, doctests

\begin{frame}
  \frametitle{Emacs FTW !}

  \begin{block}{Une configuration simplifiée}
    emacs-for-python\footnote{\tiny\url{https://github.com/gabrielelanaro/emacs-for-python}}
    \begin{itemize}
    \item Autocomplétion (pas parfaite)
    \item Coloration
    \item Interpréteur inclus
    \item Vérification syntaxique
    \end{itemize}
  \end{block}

  \begin{block}{Améliorable}
    \begin{itemize}
    \item Documentation incluse et meilleure autocomplétion avec Jedi
    \item Lancer les tests dans emacs
    \item Coloration probablement à customizer
    \end{itemize}
  \end{block}
\end{frame}

\begin{frame}
  \frametitle{Ou un autre IDE...}

  \begin{block}{Une pléthore de choix}
    \begin{itemize}
    \item Eclipse (plugin Python)
    \item NINJA-IDE Is Not Just Another IDE (Python)
    \item Sublime Text (plugins en Python)
    \item Geany, ideas, ...
    \end{itemize}
  \end{block}
\end{frame}

\begin{frame}[fragile]
  \frametitle{PyPI et pip}
  \begin{block}{One place to find them}
    \begin{itemize}
    \item Un repo central à la Maven
    \item Utilisable par TOUS (même vous)
    \item 72.000.000 de lignes de Python
    \item 33.000 packages
    \end{itemize}
    \centering
    \url{https://pypi.python.org/pypi}
  \end{block}

  \begin{block}{One command to bring them all}
    (sudo) pip install nomdupackage

    \begin{python}
import nomdupackage # fonctionne directement !
    \end{python}
  \end{block}
\end{frame}

\begin{frame}
  \frametitle{virtualenv, pour gérer les versions}

  \begin{block}{Le monde à un problème...}
    \begin{itemize}
    \item Gestion de plusieurs projets à la fois
    \item Versions de packages/Python différentes
    \end{itemize}
  \end{block}

  \begin{block}{Une solution : virtualenv}
    \begin{itemize}
    \item Environnement de travail indépendant
    \item Gestion des versions
    \item Environnement "sain" (pas de packages inutiles)
    \end{itemize}
  \end{block}
\end{frame}

\begin{frame}
  \frametitle{virtualenvwrapper}

  \begin{block}{virtualenv, un peu répétitif}
    \begin{itemize}
    \item Les environnements peuvent être dispersés sur la machine
    \item Beaucoup de tâches non automatisées
    \item Un peu verbeux
    \end{itemize}
  \end{block}

  \begin{block}{virtualenvwrapper : encore plus simple !}
    \begin{itemize}
    \item Organise tous les environnements dans un seul endroit
    \item Facilite la gestion des environnements (switch, création, ...)
    \item Extensible
    \end{itemize}
  \end{block}

\end{frame}

\begin{frame}
  \frametitle{virtualenvwrapper}

  \begin{figure}[!h]
    \centering
    \includegraphics[width=0.7\textwidth]{rubiks_cat}
    \caption{DEMO !}
    \label{fig:virtualenvwrapperdemo}
  \end{figure}

\end{frame}

\begin{frame}
  \frametitle{Documentation 101}

  \begin{block}{Le monde à un problème...}
    \begin{itemize}
    \item Processus de test souvent détaché du code
    \item Test-driven code souvent illusoire
    \item Un univers à part (syntaxe différente, méchanismes
      distincts...)
    \end{itemize}
  \end{block}

  \begin{block}{doctests : la solution simple}
    \begin{itemize}
    \item Inclure les tests à la documentation
    \item Permettre de mélanger tests et code
    \end{itemize}
  \end{block}

\end{frame}


\begin{frame}
  \frametitle{Doctest FTW !}

  \begin{figure}[!h]
    \centering
    \includegraphics[width=0.8\textwidth]{sysfail}
    \caption{DEMO !}
    \label{fig:doctestdemo}
  \end{figure}
\end{frame}


\section{Logging, Visualisation et Dév Web}
\label{sec:libraries}

\begin{frame}
  \frametitle{Le log}

  \begin{block}{Le monde a un problème...}
    \begin{itemize}
    \item Besoin de suivre l'exécution de ses algorithmes
    \item Différents besoins d'information (debug, suivi, erreurs,
      ...)
    \item Print, l'erreur du débutant
    \end{itemize}
  \end{block}

  \begin{block}{logging, la solution}
    \begin{itemize}
    \item Un module standard
    \item Facile à utiliser
    \item Très bien documenté\footnote{\url{http://docs.python.org/3.3/howto/logging}}
    \end{itemize}
  \end{block}
\end{frame}

\begin{frame}
  \frametitle{Démo time !}
  \begin{figure}[!h]
    \centering
    \includegraphics[width=0.7\textwidth]{catfail}
    \caption{Démo Logging}
    \label{fig:loggingdemo}
  \end{figure}
\end{frame}

\begin{frame}
  \frametitle{Visualisation de données}

  \begin{block}{Un besoin}
    \begin{itemize}
    \item Faire des schémas, courbes facilement
    \item Utiliser directement des données d'un algo python
    \item Diversité des besoins (ML, stats, patatoïdes, ...)
    \end{itemize}
  \end{block}

  \begin{block}{Une solution}
    Matplotlib, un module pour faire tout ça.
  \end{block}
\end{frame}

\begin{frame}
  \frametitle{Démo colorée !}

  \begin{figure}[!h*]
    \centering
    \includegraphics[width=0.7\textwidth]{flyingcat}
    \caption{Démo matplotlib}
    \label{fig:matplotlibdemo}
  \end{figure}
\end{frame}

\begin{frame}
  \frametitle{Django, pour le web}

  \begin{block}{Un framework simple}
    \begin{itemize}
    \item Tuto
      efficace\footnote{\url{https://docs.djangoproject.com/en/dev/intro/tutorial01/}}
    \item ORM, sqlite, ...
    \item Tuto efficace\footnote{\url{http://www.creativebloq.com/netmag/get-started-django-7132932}}
    \item Séparation des scripts (bdd, interface, routes, ...)
    \item Tutos efficaces\footnote{\url{https://code.djangoproject.com/wiki/Tutorials}}
    \end{itemize}
  \end{block}
\end{frame}

% matplotlib, logging, django
% http://docs.python.org/3.3/howto/logging

\section{Conclusion et Pointeurs}
\label{sec:conclusion}

\begin{frame}
  \frametitle{En résumé}

  \begin{block}{Python, un langage...}
    \begin{itemize}
    \item Idéal pour la plupart des tâches
    \item Avec un nombre de modules impressionnant
    \item Avec une syntaxe agréable
    \item Qui laisse un peu plus de champ libre aux développeurs
    \end{itemize}
  \end{block}

  \begin{block}{Mais...}
    \begin{itemize}
    \item Qui laisse un peu plus de champ libre aux développeurs
    \item Qui n'est pas encore totalement passé dans sa v3
    \end{itemize}
  \end{block}

\end{frame}

\begin{frame}
  \frametitle{Les ressources}

  \begin{block}{Si j'ai une question...}
    \begin{itemize}
    \item La documentation
      Python\footnote{\url{http://docs.python.org/3/}}
    \item SO pour les questions techniques (pas trop pour le style)
    \item IRC (freenode)
    \end{itemize}
  \end{block}

  \begin{block}{Envie de voir du beau code et d'apprendre}
    \begin{itemize}
    \item Idiomes sympas\footnote{http://sahandsaba.com/thirty-python-language-features-and-tricks -you-may-not-know.html}
    \item Tuto interactif\footnote{\url{https://pythonmonk.com/}}
    \item CodeAcademy ?
   \end{itemize}
  \end{block}
\end{frame}


\begin{frame}
  \frametitle{Les ressources 2}

  \begin{block}{Des vidéos}
    \begin{itemize}
    \item
      NeckBeard\footnote{\url{https://www.neckbeardrepublic.com/screencasts/}}
      : excellents screencast (Sphinx, PyPI, Django, Profiling,
      itérables, ...)
    \item Les conférences
      Python\footnote{\url{http://pyvideo.org/}}. Parfois techniques...
    \end{itemize}
  \end{block}
\end{frame}

\begin{frame}
  \frametitle{The End !}

  \begin{figure}[!h*]
    \centering
    \includegraphics[width=0.7\textwidth]{byecat}
    \caption{Fini !}
    \label{fig:end}
  \end{figure}
\end{frame}

% beardneck, SO, Sphinx, pyvideos



\end{document}