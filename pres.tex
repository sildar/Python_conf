\documentclass{beamer}

%\usetheme{Warsaw}
\usetheme{Frankfurt}
\beamertemplatenavigationsymbolsempty

\usepackage[utf8]{inputenc}
\usepackage[francais]{babel}
\usepackage{hyperref}
\usepackage{amsmath}
\usepackage{graphicx}
\graphicspath{{./img/}}
\DeclareGraphicsExtensions{.png, .jpeg, .jpg}


\renewcommand*\thesection{\arabic{section}}


\AtBeginSection[]{%
  \begin{frame}<beamer>
    \frametitle{Outline}
    \tableofcontents[sectionstyle=show/hide,subsectionstyle=hide/show/hide]
  \end{frame}
  \addtocounter{framenumber}{-1}
}



\title{Python : Une Introduction}
\institute{ASCII}
\author{Rémi Bois}
\date{\today}

\begin{document}

\begin{frame}
  \maketitle
\end{frame}

\section{Introduction}
\label{sec:intro}


\begin{frame}
  \frametitle{Objectifs de la présentation}
% Do-s & Don't-s
\end{frame}

\begin{frame}
  \frametitle{Python, Kézako ?}
% langage de programmation/script
% date d'apparition
% parenté avec les Monthy Pythons
% Parenté avec Perl "many ways to do it" -> "one obvious way"
\end{frame}

\begin{frame}
  \frametitle{Encore un langage de niche ?}
% Grands projets (voir django, ...)
% Place dans l'open source
\end{frame}

\begin{frame}
  \frametitle{Sa particularité ?}
% Lisibilité (quote Guido)
% Open
% Très bon dans certains domaines (nltk, numPy, ...)
% Adossé au C (ahah)
\end{frame}

\begin{frame}
  \frametitle{Sa version ...?}
% Python 2 vs Python 3
% problème des modules non portés
% problèmes sémantiques (/, print, ...)
\end{frame}

\section{Syntaxe}
\label{sec:syntax}

\begin{frame}
  \frametitle{L'avènement des tabulations}
% Objectif (lisibilité)
% Un exemple Java/C++ -> Python
\end{frame}

\begin{frame}
  \frametitle{Le typage en Python}
% Non verbosité
% nommage des variables plutôt que typage
\end{frame}

\begin{frame}
  \frametitle{Les conditions}
% If, While, For
% The Pythonic Way of doing it (lists)
\end{frame}

\begin{frame}
  \frametitle{Les fonctions}
% Définition, appel
% Paramètres supplémentaires
\end{frame}

\section{Les listes et les paires}
\label{sec:lists}

\begin{frame}
  \frametitle{Les tuples présents, comme dans tout bon langage}
% taunt Java
% définition, accès, swap de variables...
\end{frame}

\begin{frame}
  \frametitle{Les listes comme on les connaît}
% définition, parcourt, fonctions principales
\end{frame}

\begin{frame}
  \frametitle{List comprehension}
% exemple simple avec zip (en lien avec les tuples)
\end{frame}

\begin{frame}
  \frametitle{List comprehension avancée}
% nested lists, fontion sur les éléments
\end{frame}

\section{Une approche fonctionnelle}
\label{sec:functionnal}

\begin{frame}
  \frametitle{Les maps, reduce, filter}
% fonctions avec exemple
\end{frame}

\begin{frame}
  \frametitle{Les lambda expressions}
% exemple
\end{frame}

\section{Exceptions \& co}
\label{sec:except}

\begin{frame}
  \frametitle{Les exceptions}
% Better ask for forgiveness than for permission
% exemple
\end{frame}

\begin{frame}
  \frametitle{Breaks \& Continues}
% exemple
\end{frame}




\end{document}